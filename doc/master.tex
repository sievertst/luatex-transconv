\documentclass{ltxdockit}

%%%%%%%%
% Meta %
%%%%%%%%
\title{The \sty{transconv} Package}
\author{Sievert\ von\ St\"{u}lpnagel}
\date{\today}

%%%%%%%%%%%%
% Settings %
%%%%%%%%%%%%
\usepackage{luapackageloader} % for using custom search paths
  \newcommand{\LuaDir}{../lua}
  \directlua{package.path = "\LuaDir/?.lua;\LuaDir/?/init.lua;" .. package.path}

\usepackage{luatexja-fontspec}
\usepackage{polyglossia}
  \setmainlanguage[variant=uk]{english}

\usepackage{booktabs}
\usepackage{makecell}

\usepackage{luacolor}

\usepackage{hologo}

\usepackage[%
    scheme={nan.tailo},
    scheme={nan.poj},
    scheme={nan.bp},
  ]{transconv}

\usepackage{listings}

\usepackage[inline]{enumitem}

\usepackage{array}

\usepackage{csquotes}

\hypersetup{
    unicode         = true,
    breaklinks      = true,
    hidelinks       = true,
    colorlinks      = true,
    citecolor       = blue,
    urlcolor        = blue,
    linkcolor       = blue,
    % pdf meta data
    pdfauthor       = {Sievert von Stülpnagel},
    pdftitle        = {The transconv Package},
    pdfsubject      = {transconv},
    pdflang         = english,
    % pdf reader settings
    pdffitwindow    = true,
    pdfstartview    = FitH,
    pdfnewwindow    = true,
  }

\usepackage[%
    xindy,
    record,
    abbreviations,
    style = super,
  ]{glossaries-extra}

\usepackage{silence}
  % \WarningFilter[fontfilter]{latexfont}{Font shape}
  % \WarningFilter[fontfilter]{latexfont}{Some font shapes}
  \WarningFilter{todonotes}{The length marginparwidth is less than}
  \WarningFilter{glossaries-extra}{No entries defined}
  \WarningFilter{forest}{Baseline node}
  \ActivateWarningFilters[fontfilter]

\newcommand*\subsubsubsection[1]{\vspace{1em}\noindent\textsf{\textbf{#1}}}

\newcommand{\textvbaraccent}[1]{#1\symbol{"030D}}

%%%%%%%%%
% Fonts %
%%%%%%%%%
\setmainfont{Libertinus Serif}
% \setmainjfont{BabelStone Han}

%%%%%%%
% TOC %
%%%%%%%
\AtBeginToc{\setcounter{tocdepth}{1}}
\AtEndToc{\setcounter{tocdepth}{5}}

%%%%%%%%%%%%%%%
% Source Code %
%%%%%%%%%%%%%%%
\lstset{language=[5.3]Lua}


\begin{document}

\maketitle

\tableofcontents
\listoftables

\section{Introduction}\label{int}

\subsection[About]{About \sty{transconv}}

\subsection{License}

\section{Package Options}

\begin{optionlist}
  \optitem{scheme}{\prm{lang}.\prm{scheme}}

  This option enables support for the transliteration scheme \prm{scheme} for
  the language \prm{lang} by setting up the matching converter in Lua and
  defines the commands \sty{\textbackslash{}\prm{scheme}convert},
  \sty{\textbackslash{}\prm{scheme}font} and
  \sty{\textbackslash{}to\prm{scheme}}. It will also ensure that the
  \sty{\textbackslash{}to\prm{lang}} and \sty{\textbackslash\prm{lang}font}
  commands exist by defining them the first time you define a scheme for a new
  language. Loading more schemes for the same language after that will not
  change that default; if you want to change it, use the \sty{defaultscheme}
  option or the \sty{\textbackslash{}TransconvMakeDefaultScheme} command.

  In order to successfully load a scheme, Transconv has to be able to find
  a file \prm{lang}/\prm{scheme}.lua in its lua folder which returns the
  converter. The option can be used more than once to import multiple schemes.

  \optitem{scheme}{\{\prm{lang1}.\prm{scheme1},
  \prm{lang2}.\prm{scheme2},\dots\}}

  Imports multiple schemes at once. Equivalent to a repeated use of
  \opt{scheme}.

  \optitem{defaultscheme}{\prm{lang}.\prm{scheme}}

  Defines the default scheme for the language \prm{lang} in case more than one
  scheme is added for the same language. Can be used more than once to define
  defaults for multiple languages. In case of repeated use for the same
  language, later schemes override earlier ones.

  TD

  \optitem{defaultscheme}{\{\prm{lang1}.\prm{scheme1},
  \prm{lang2}.\prm{scheme2},\dots\}}

  Defines default schemes for multiple languages at once. Equivalent to a
  repeated use of \opt{defaultscheme}. If the same language appears more than
  once, later schemes override earlier ones.

  TD

\end{optionlist}

\section{Basic Interface}

All commands in this section are robust and can therefore be used in
expansion-only contexts (e.\,g.\ headings or footnotes) without danger. However
the list of commands you can use within the conversion argument is unfortunately
very limited. This is because such commands would be expanded before the string
is sent to the converter, which can easily result in an error.

As a rule of thumb, if your command is a simple macro which expand to nothing
but text like in the following example, it should be fine:

\begin{ltxcode}
  \newcommand\mystring{Zhe4 mei2you3 wen4ti2.}
  \topinyin{\mystring}
\end{ltxcode}

However commands which change some settings (e.\,g.\ the font) will likely
cause an error. For example both of the following will not compile:

\begin{ltxcode}
  \topinyin{\emph{Zhe3} wu2fa3 bian1yi4.}
  \topinyin{{\itshape Zhe3} ye3 bu4 xing2.}
\end{ltxcode}

\noindent

\subsection{Formatted Conversion Commands}

\begin{ltxsyntax}
  \cmditem{tolang}{text}

  where \sty{lang} is a language declared via
  \sty{\textbackslash{}TransconvUseScheme} or the package option \opt{scheme}.
  This command will convert \prm{text} to the default scheme for \sty{lang}
  (either the one defined by \opt{defaultscheme} or else the first scheme
  defined for \sty{lang}), using the formatting defined in
  \sty{\textbackslash{}langfont}.

  In most cases, this should be the go-to conversion command to use in your text
  as it abstracts both the scheme itself as well as its formatting away from the
  content.

  \cmditem{toscheme}{text}

  where \sty{scheme} is a scheme declared via
  \sty{\textbackslash{}TransconvUseScheme} or the package option \opt{scheme}.
  This command will convert \prm{text} to the \sty{scheme} using the formatting
  defined in \sty{\textbackslash{}schemefont}.

  Use this command to access specific schemes if you need to import multiple
  ones for a single language (for example if the schemes themselves are a topic
  of discussion). Note that \sty{\textbackslash{}toscheme} is provided for all
  imported languages, including the default one. If you only intend to use a
  single theme for a given language, use \sty{\textbackslash{}tolang} instead to
  avoid hardcoding the scheme into your text.

\end{ltxsyntax}

\subsection{Unformatted Conversion Commands}

\begin{ltxsyntax}
  \cmditem{langconvert}{text}

  where \sty{lang} is a language declared via
  \sty{\textbackslash{}TransconvUseScheme} or the option \opt{scheme}. This
  command will convert \prm{text} to the default scheme for \sty{lang} (either
  the one defined by \opt{defaultscheme} or else the first scheme defined for
  \sty{lang}), but without any formatting. In other words,
  \sty{\textbackslash{}langconvert} abstracts away the transcription scheme but
  not the formatting.

  It is recommended that you use this command only if the formatting hooks
  provided by \sty{transconv} are insufficient for your purposes, and only to
  define a custom macro rather than in the actual text in order to maintain the
  separation of content and presentation.

  \cmditem{schemeconvert}{text}

  where \sty{scheme} is again a scheme declared via
  \sty{\textbackslash{}TransconvUseScheme} or the option \opt{scheme}.

  This command merely constitutes a wrapper around the Lua converter without any
  \TeX{} formatting. Therefore, using this command means you hard-code both the
  transcription scheme and the way it is formatted.  For this reason, it is
  highly recommended that you use this command only if the formatting hooks
  provided by \sty{transconv} are insufficient for your purposes, and even then
  only to define a custom macro rather than in the actual text in order to
  maintain the separation of content and presentation.

\end{ltxsyntax}

\subsection{Output Formatting}

\begin{ltxsyntax}
  \cmditem{langfont}

  where \sty{lang} is a language declared via
  \sty{\textbackslash{}TransconvUseScheme} or the option \opt{scheme}.

  This command sets the font formatting for the output of
  \sty{\textbackslash{}tolang}. By default it expands to
  \sty{\textbackslash{}itshape}.

  \cmditem{schemefont}

  where \sty{scheme} is a transcription scheme declared via
  \sty{\textbackslash{}TransconvUseScheme} or the option \opt{scheme}.

  This command sets the font formatting for the output of
  \sty{\textbackslash{}tolang}. By default it expands to nothing.

\end{ltxsyntax}

\section{Configuration}

\begin{ltxsyntax}
  \cmditem{TransconvUseScheme}{schemes}

  Sets one or more new scheme(s) up for later usage. Equivalent to using the
  \sty{scheme} package option. As with the option, each scheme should be specified as
  \sty{\prm{lang}.\prm{scheme}} and multiple schemes can be set up at once by
  stringing them together with commas.

\end{ltxsyntax}

\section{Custom Schemes and Languages}

\subsection{Adding a New Transliteration Scheme for an Existing Language}

Adding a new transliteration scheme involves a little bit of Lua programming,
though \sty{transconv} tried to make it as painless as possible.

As a first step, navigate to the directory where the Transconv lua files are
found (the exact location depends on where you installed it, but it should be
called transconv and contain a file named init.lua.

Once in this directory add a new file \prm{scheme}.lua in the folder \prm{lang}.
The name of the file will be the scheme name which you have to use to load the
scheme later. Open the file in the text editor of your choice and define a new
scheme:

\begin{lstlisting}
local MyScheme = Converter:new{
  -- load raw scheme settings for your language
  raw = require(transconv.path_of(...)..".raw"),

  -- settings variables are going to go here
}
\end{lstlisting}

You can use any name you want instead of MyScheme. This name is only used to
refer to the scheme within the file itself; it has no consequence outside.

At the end of the file, return the scheme:

\begin{lstlisting}
return MyScheme
\end{lstlisting}

Technically speaking you are now set; you have successfully defined a new scheme
and should be able to load it by passing \sty{\prm{lang}.\prm{scheme}} to
\sty{transconv}'s \sty{schemes} option or the
\sty{\textbackslash{}TransconvUseScheme} macro (make sure to use the
folder and file name, \emph{not} the name you used inside the Lua file).

However since you didn't specify any settings, \sty{transconv} will use the
default ones -- which result in no changes to the input at all. To get your
scheme to do something useful, you have to override these settings. This can be
done by adding member variables and/or functions to your scheme.

\subsubsection{Default Member Variables of Schemes}

Member variable settings don't change the conversion process itself but merely
provide resources which the scheme uses during this process. For minor changes,
you usually only need to set a member variable and can leave the algorithm
itself alone. The following variables are available to you by default (do not
forget to add a comma after each member or Lua will get confused!):

\paragraph{\sty{raw}}

This variable tells your scheme which raw (input) scheme to use. This should
pretty much always be set to the following:

\begin{lstlisting}
raw = require(transconv.path_of(...)..".raw"),
\end{lstlisting}

This will cause Lua to load the scheme from the \sty{raw.lua} file in the same
folder.

\paragraph{\sty{rep\textunderscore{}strings}}

This variable is probably going to be your best friend because it is what tells
your scheme to replace certain letter sequences with others. It contains a
comma-separated list of string pairs surrounded by curly braces. Each string
(letter sequence) should be surrounded with (double or single) quotes and the
two items of each pair should be separated with a comma also like so:

\begin{lstlisting}
rep_strings = {
  {"c", "k"}, {"ts", "ch"},
},
\end{lstlisting}

During conversion, your scheme will look at each of the pairs, find all
instances of the first item in your input and replace it with the second item.
For example, the above settings will cause it to replace every \enquote{c} with
\enquote{k} and every \enquote{ts} with \enquote{ch} (note that by
default, the search is case-insensitive, so \enquote{C}, \enquote{Ts},
\enquote{TS} and \enquote{tS} will also be replaced).

Be aware that the replacements are executed in the order in which you defined
them, which means earlier rules can feed into later ones if you're not careful.
For example, if we swap the pairs above around:

\begin{lstlisting}
rep_strings = {
 {"ts", "ch"}, {"c", "k"},
},
\end{lstlisting}

Then the first rule will first replace every \enquote{ts} with \enquote{ch}, but
then the second rule will replace the \enquote{c} with \enquote{k} and you end
up with the possibly unexpected \enquote{kh}. So if you get surprising
replacements, have a look at the order of rules and check if any might be
feeding into later ones.

The second thing you have to pay attention to is that certain characters
have a special meaning in Lua, so in order for your scheme to use their literal
values, you have to escape them. You can find the full list in
\autoref{tab:luaescape}.

\begin{table}[ht]
  \centering
  \begin{tabular}{>{\ttfamily}c >{\ttfamily}c}
    \multicolumn{1}{c}{\textbf{literal character}}
      & \multicolumn{1}{c}{\textbf{escape sequence}} \\
    \textbackslash    & \textbackslash\textbackslash \\
    \textquotesingle  & \textbackslash\textquotesingle \\
    "                 & \textbackslash" \\
    .                 & \%. \\
    -                 & \%- \\
    *                 & \%* \\
    \%                & \%\% \\
    (                 & \%( \\
    )                 & \%) \\
    +                 & \%+ \\
    ?                 & \%? \\
    \textasciicircum  & \%\textasciicircum \\
    \lbrack           & \%\lbrack \\
    \$                & \%\$ \\
  \end{tabular}
  \caption{Lua string escape sequences}
  \label{tab:luaescape}
\end{table}

So if, for example, you want to replace all instances of \enquote{aa} with
\enquote{\^{a}}, the correct rule would be:

\begin{lstlisting}
{"aa", "\\%^{a}"}
\end{lstlisting}

\paragraph{sb\textunderscore{}sep}

Use this variable to define a separator if your output scheme requires one to be
inserted between different syllables (for example the Wade-Giles scheme for
Mandarin Chinese has syllables separated with a hyphen). The separator is not
inserted before a space or special characters. The default setting is an empty
string.

\paragraph{tone\textunderscore{}markers}

For tonal languages which mark the tones with diacritics, list all tones as
integer keys with the macro name for the marker as a value (without the
leading backslash). Tones which do not have such a marker should be marked as
\sty{false}. For example the correct setting for Hanyu Pinyin would
be:\footnote{The elements 0 and 5 are both set false so the user can use either
integer to mark the neutral tone. The single quote for the second tone macro
name has to be escaped because Lua would otherwise take it as a special
character (cf. \autoref{tab:luaescape}).}

\begin{lstlisting}
tone_markers = {
  [0] = false, [1] = "=", [2] = "\'", [3] = "v", [4] = "`", [5] = false,
},
\end{lstlisting}

The converter will then end up replacing the input \sty{a1} with
\sty{\textbackslash=\{a\}}, \sty{a2} with \sty{\textbackslash'\{a\}} etc.

In case your tone numbers are consecutive integers starting with 1, you can also
simply list the marker strings without explicitly stating the index. So if we
disallow using \sty{0} for the neutral tone, the above could also be simplified
to:

\begin{lstlisting}
tone_markers = { "=", "\'", "v", "`", false},
\end{lstlisting}

\paragraph{second\textunderscore{}rep\textunderscore{}strings}

For tonal languages it may occasionally be necessary to do a second round of
string replacement after it is already decided where the tone marker should go.
Use this variable for this purpose. It works the same way as
\sty{rep\textunderscore{}strings}.

\paragraph{no\textunderscore{}tones}

For tonal languages, this variable can be set to either \sty{true} or
\sty{false}. If \sty{true}, your scheme will simply delete all tones from your
output. This allows you to first write your document with tones but then turn
them off if your publisher wants tones to be omitted.

\subsubsection{Adding and Overriding Methods}

If the variable settings are not sufficient to produce the intended result, you
can override the default functions of your scheme or add your own ones to
supplement them. You may also choose to remove unneeded default functions to get
a slight boost in performance if you find conversion is too slow. However, this
requires at least a basic understanding of Lua to write the new functions. I
will therefore assume for this section that you know how to define a function
and add it to a table.

\subsubsubsection{Overriding Default Methods}

Any new transcription scheme will provide you with the following default methods
which you can override by simply defining your own custom version and adding it
to your scheme table.

\paragraph{convert(self, input)}

This is the central function of your scheme. It must always be present because
this is the function which \sty{transconv} will call when you use
\sty{\textbackslash{}tolang} or a similar command in your document.

By default it will:

\begin{enumerate}
  \item split the input into syllables by calling the associated \sty{raw}
    scheme's \sty{split\textunderscore{}sbs} method,
  \item check each syllable if it is a valid syllable in the associated
    \sty{raw} scheme by calling its
    \sty{is\textunderscore{}valid\textunderscore{}sb} method. If not, the
    syllable is funneled directly into the output without any further
    processing.\footnote{This allows you to use foreign words in the input.}
  \item For valid syllables, it checks if there is already a cached conversion
    result. If not, it will call the
    \sty{to\textunderscore{}target\textunderscore{}scheme} function which
    handles the actual conversion, store the result in the cache and then funnel
    it to the output.
  \item After all syllables have been processed, they are joined back together
    using the \sty{join\textunderscore{}sbs} method and the result is returned.
\end{enumerate}

\paragraph{to\textunderscore{}target\textunderscore{}scheme(self, syllable)}

The most basic conversion function. It will:

\begin{enumerate}
  \item call the associated \sty{raw} scheme's
    \sty{get\textunderscore{}sb\textunderscore{}and\textunderscore{}tone} method
    to separate potential tone digits at the end and store them separately,
  \item call the \sty{do\textunderscore{}str\textunderscore{}rep} method using
    the \sty{rep\textunderscore{}strings} member variable to execute string
    replacements,
  \item call the \sty{place\textunderscore{}tone\textunderscore{}digit} method
    to identify the correct letter which should carry the marker and insert the
    tone digit after it,
  \item call the \sty{do\textunderscore{}str\textunderscore{}rep} method
    again, but this time using the
    \sty{second\textunderscore{}rep\textunderscore{}strings} member variable for
    secondary replacements,
  \item check if the \sty{no\textunderscore{}tones} variable is set to
    \sty{true}. If so, it simply deletes any digits from the string. Otherwise,
    it calls \sty{add\textunderscore{}tone\textunderscore{}markers} to replace
    the digits with the correct tone markers,
  \item return the end result.
\end{enumerate}

\paragraph{do\textunderscore{}str\textunderscore{}rep(self, syllable,
list\textunderscore{}of\textunderscore{}replacements)}

This function will perform the actual string replacements according to the
provided list. It assumes that the list is of the same form as
\sty{rep\textunderscore{}strings} above, i.\,e.\ a table of tables, where each
of the inner tables contains exactly two strings (original and replacement). It
will then loop over the outer table and for each member table:

\begin{enumerate}
  \item convert both the input syllable and the search string to lower so case
    is ignored,
  \item search the lower-case syllable for instances of the search string. If it
    finds any, it then:
    \begin{enumerate}
      \item checks the case of the first letter in the match for case. If the
        former is lowercase, it will assume all lower case.  If it is upper and
        the following one is lower, it will assume title case. If both are
        upper, it will assume all upper case.
      \item performs the string replacement in the appropriate casing,
    \end{enumerate}
\end{enumerate}

After it has finished the loop it returns the end result.

\paragraph{place\textunderscore{}tone\textunderscore{}digit(self, syllable,
tone)}

For tonal languages, this method is responsible for placing the tone digit back
into the syllable at the correct position. If some other letter will carry the
tonal information (typically using a diacritic, but possibly with other means,
e.\,g.\ reduplication, replacement with another letter etc), this method should
identify that letter and place the raw digit behind it. Note that this function
only handles placement; the conversion into the correct output form will be
handled later by the \sty{add\textunderscore{}tone\textunderscore{}marker}
function.

By default, this function simply adds the tone digit back to the end of the
string, so if your scheme requires a different behaviour, you will have to
override this method.

\paragraph{add\textunderscore{}tone\textunderscore{}marker(self, syllable)}

This function looks for digits in the input syllable. If it finds one, it looks
up in the \sty{tone\textunderscore{}markers} variable to find the name of the
replacement macro. It then wraps the preceding letter in that macro and deletes
the digit. For example, if the \sty{tone\textunderscore{}markers} variable
contains the value \sty{"="} at index \sty{1} and your input string is
\sty{a1ng}, then this function will return \sty{\textbackslash=\{a\}ng}.

Note that this function should only be used for tones marked with diacritics. If
the tone is marked in another way (e.\,g.\ reduplication of a letter), add
corresponding replacement rules in the
\sty{second\textunderscore{}rep\textunderscore{}strings} instead.

\subsubsubsection{Adding Your Own Methods}

To add your own method to the conversion process, you have to take two steps:

\begin{enumerate}
  \item Implement your Method and add it to your scheme table. If it needs
    access to any other member variables or methods, make sure to pass a
    reference to your table as the first argument (either by using Lua's colon
    syntax or as an explicit argument).
  \item Override one of the default methods and have it call your custom method
    at the appropriate step with the appropriate arguments.
\end{enumerate}

\subsection{Adding a New Language}

To add a new language for which transcription schemes can be implemented, first
add a new folder where \sty{transconv} can find it. The folder name will be the
language name you will have to use when loading schemes from your document
later, so make sure you choose something unique and easy to memorise. The
default languages use the ISO 693-3 abbreviations and it is strongly suggested you
stick to the same convention. If you need to specify a certain subgrouping
within a bigger language variety for which no ISO abbreviation has been coined,
specify the location after a dash, for example \sty{jpn-kyoto} for
Ky\={o}t\={o} Japanese.

As a second step, you will have to choose a raw input scheme. This scheme should
fulfil the following criteria:

\begin{itemize}
  \item It must contain enough information to convert to any intended target
    scheme. More specifically, if any feature is reflected even in just one
    possible target scheme, it must be reflected in the raw scheme also,
    otherwise conversion to that target scheme will be incorrect. For example,
    the Japanese Hiragana characters じ and ぢ are pronounced exactly the same
    (ji) and most transcription schemes spell them the same as well. However
    a select few -- most notable the Nihon-shiki scheme -- do reflect the
    difference in Kana spelling. Therefore, the raw scheme has to make the
    distinction as well to allow for accurate conversion to those
    schemes.\footnote{In this case, I decided to follow Nihon-shiki and spell
    them according to their original phonetic series: \sty{zi} for じ and
    \sty{di} for ぢ.}
  \item It is strongly suggested that the raw scheme should not make use of
    non-ascii characters. The reason for this is that Lua makes use of your
    computer's locale for certain aspects of string handling. As a result, if you
    use non-ascii characters, the code may or may not work as you expect it on
    your machine. But even if it does work for you, it might not do so on
    a machine in a different locale.
\end{itemize}

Once you have decided on your raw scheme, add a file called \sty{raw.lua} to
your language folder.\footnote{You don't have to use this name but I suggest
you follow the convention. If you don't, any authors of new target schemes
will have to use your different name when importing the raw scheme to their
file, which could confuse especially unexperienced Lua users.} Document your raw
scheme in a comment section at the top.\footnote{If you are using an existing
scheme or a slightly modified version of it, you can make this very brief by
referencing it, e.\,g.: \enquote{Pinyin, just with tone numbers at the end of
each syllable instead of diacritics}.}

Below that, add a table for the raw scheme and return it:

\begin{lstlisting}
local raw = Raw:new{
}

return raw
\end{lstlisting}

Then populate your table with settings and methods. The following are provided
by default:

\subsection{Default Raw Scheme Member Variables}

\paragraph{cutting\textunderscore{}markers}

A list of strings which can be used to define borders between syllables when
splitting up an input strings using the \sty{split\textunderscore{}sbs} method.
If not empty, your raw scheme will scan the input string and make a cut whenever
it finds one of the strings in this list. By default, the list is empty.

\subsection{Default Raw Scheme Methods}

\paragraph{get\textunderscore{}sb\textunderscore{}and\textunderscore{}tone(self, input)}

Expects a single syllable with information on the tone in it. Returns the raw
syllable (without tone information) and an integer representing the tone. By
default it will always simply return the input string and 0. If your language is
non-tonal, simply ignore this method.

\paragraph{is\textunderscore{}valid\textunderscore{}sb(self, input)}

Use this method to test if the passed string is a valid syllable in your input
string. If it returns \sty{false} for a given string, \sty{transconv} will
simply not make any changes to it. By default it always returns \sty{true}.

\paragraph{split\textunderscore{}sbs(self, input)}

Used to split the input into syllables. If your language does not require this
(like Japanese for instance), you can override this method to simply return a
list with the input string as its only member.

By default it will:

\begin{enumerate}
  \item Check if the \sty{cutting\textunderscore{}markers} variable is empty. If
    it is, it will simply make a cut before each non-word character (special
    characters or whitespace). Otherwise it will search the input for any
    strings in the list and make a cut before any matches.
  \item Either way, it puts all syllables in a list and returns it.
\end{enumerate}

\section{Package Code}

\subsection{The \sty{transconv.sty} file}

\subsection{The Lua code}


\tonan{li2 ho2}
\topoj{peh8-ue7-ji7}

\end{document}
